\documentclass{article}
\usepackage{geometry}
\usepackage{amsmath}
\usepackage{color}
\usepackage{xcolor}
\definecolor{keywordcolor}{rgb}{0.8,0.1,0.5}
\usepackage{listings}
\usepackage{xcolor}
\usepackage{cite}
\definecolor{mygreen}{rgb}{0,0.6,0}
\definecolor{mygray}{rgb}{0.5,0.5,0.5}
\definecolor{mymauve}{rgb}{0.58,0,0.82}
\lstset{ %
backgroundcolor=\color{white},      % choose the background color
basicstyle=\footnotesize\ttfamily,  % size of fonts used for the code
columns=fullflexible,
tabsize=4,
breaklines=true,               % automatic line breaking only at whitespace
captionpos=b,                  % sets the caption-position to bottom
commentstyle=\color{mygreen},  % comment style
escapeinside={\%*}{*)},        % if you want to add LaTeX within your code
keywordstyle=\color{blue},     % keyword style
stringstyle=\color{mymauve}\ttfamily,  % string literal style
frame=single,
rulesepcolor=\color{red!20!green!20!blue!20},
% identifierstyle=\color{red},
language=c++,
}
\lstset{breaklines}
\lstset{extendedchars=false}

\author{Bao Jinge}
\title{Solutions for Week 1}
\date{}

\begin{document}
	\maketitle
	\section{}
		As two kinds of buses stop at the bus stop every half hours, so we can just image a scene with an hour span. In this scene(which time interval is $[0,60)$), every kind of buses can stop only twice. Here, we suppose without loss of generality as following\\
		1. the first kind of buses(go to southwards) will stop at timestep $a_1$ and $a_2$ and the second buses(go to northwards) will stop at timestamp $b_1$ and $b_2$. $a_1, a_2, b_1, b_2\in[0,60)$.\\
		2. $a_1 < b_1$, so $a_2 < b_2$ too.\\
		3. Every month has only 30 days.\\
		Here are equations
		$$\frac{(a_1 - 0) + (a_2 - b_1) + (60 - b_2)}{(b_1 - a_1) + (b_2 - a_2)} = \frac{27}{3}$$
		$$b_1 - a_1 = b_2 - a_2$$
		$$(60 - b_2) + (a_1 - 0) = a_2 - b_1$$
		$$a_2 - a_1 = b_2 - b_1 = 30$$
		We got answer
		$$b_1 - a_1 = b_2 - a_2 = 27$$
		which means that the second kinds of bus(go south) always arrives the stop 27 minutes after the first kinds of bus(go north). So the person only only go south with probility of 27/30 and go north with probility of 3/30. Thus he visits his parents only three times per month.

	\section{}
	(a)\\
	1.When $n = 2^k$,\\
		$$randbit(n) = \sum_{i = 0}^{k-1}2^irandbit()$$
		2.When n is not a power of 2, we can write n as $n = \sum_{i=0}{} = a_i * 2^i$, which $a_i \in {0,1}$
		So we get
		$$randbit(n) = \sum_{i = 0}{}a_i \cdot randbit(2^i)$$
	(b) In C++, we can get $randbit()$ as follows
		\begin{lstlisting}
		// C++ code
		int randbit() {
			srand(time(NULL))
			int random = rand() % 2; //rand() is included in cstdlib
			return random;
		}
		\end{lstlisting}
		After running code, we got the length of the longest streak in my array a is 7.
	\section{}
		After running codes, we got the minimum cut size is 3, and i need nearly 25 iterations of the subroutine to detect this. In codes, i use Disjoint Set Union algorithm. Some codes are as follows
	\begin{lstlisting}
		while (true) {
		int edge[MAXV][MAXV];
		vector<int> v;
		v.clear();
		for (int i = 0; i < nV; i++) {
			for (int j = 0; j < nV; j++) {
				edge[i][j] = _edge[i][j];
			}
		}
		for (int i = 0; i < nV; i++) {
			fa[i] = i;
		}
		for (int i = 0; i < nV; i++) {
			v.push_back(i);
		}
		while(v.size() > 2) {
			int index_i = rand() % v.size();
			int index_j = rand() % v.size();
			int fa_i = getfa(v[index_i]);
			int fa_j = getfa(v[index_j]);
			if(fa_i != fa_j) {
				if (edge[fa_i][fa_j] >= 1) {
					if (fa_i != fa_j) {
						fa[fa_j] = fa_i;
						for (int k = 0; k < nV; k++) {
							fa[k] = getfa(fa[k]);
						}
						for (int k = 0; k < nV; k++) {
							edge[fa_i][k] = edge[k][fa_i] = edge[fa_i][k] + edge[fa_j][k];
						}
						edge[fa_i][fa_i] = 0;
						v.erase(v.begin() + index_j);
					}
				}
			}
		}	
//		cout << "the minimum cut size:" << edge[fa[v[0]]][fa[v[1]]] << endl;
//		cout << "iterations of the subroutine need to detect this answer" << ++iter << endl;
		min_cut_size = min(min_cut_size, edge[fa[v[0]]][fa[v[1]]]);
	\end{lstlisting}
\end{document}


\bibliographystyle{plain}
\bibliography{ref}
