\documentclass{article}
\usepackage{geometry}
\usepackage{amsmath}
\usepackage{color}
\usepackage{xcolor}
\usepackage{amssymb}
\definecolor{keywordcolor}{rgb}{0.8,0.1,0.5}
\usepackage{listings}
\usepackage{xcolor}
\definecolor{mygreen}{rgb}{0,0.6,0}
\definecolor{mygray}{rgb}{0.5,0.5,0.5}
\definecolor{mymauve}{rgb}{0.58,0,0.82}
\lstset{ %
backgroundcolor=\color{white},      % choose the background color
basicstyle=\footnotesize\ttfamily,  % size of fonts used for the code
columns=fullflexible,
tabsize=4,
breaklines=true,               % automatic line breaking only at whitespace
captionpos=b,                  % sets the caption-position to bottom
commentstyle=\color{mygreen},  % comment style
escapeinside={\%*}{*)},        % if you want to add LaTeX within your code
keywordstyle=\color{blue},     % keyword style
stringstyle=\color{mymauve}\ttfamily,  % string literal style
frame=single,
rulesepcolor=\color{red!20!green!20!blue!20},
% identifierstyle=\color{red},
language=c++,
}
\lstset{breaklines}
\lstset{extendedchars=false}

\author{Bao Jinge}
\title{Solutions for Week 3}
\date{}

\begin{document}
	\maketitle
	\section{}


	\section{}
	Calculate the probability of m balls into n disctinct bins. Let $p'$ be a collection of probability of m bins selected randomly from n bins. 
	$$p' = \{p'_1, p'_2,...,p'_m\} \subsetneqq \{p_1, p_2, p_3,...,p_n\}$$
	$$\binom{n}{m}\prod_{j=1}^{m}{p'_j} < \frac{1}{2}$$
	According to Inequality of Arithmetic and Geometric means,
	we got 
	$$\binom{n}{m}\prod_{j=1}^{m}{p'_j} \leq \binom{n}{m}{(\frac{\sum_{j=0}^{m}{p'_j}}{m})^m}$$
	when all $p'_j$ are same, inequality above becomes equanlity.
	As for any randomized selections, inequations above must be true and we must get maximum, there must be that $p'_j$ always the same whatever selections we choose. And we have $\sum_{i=1}^{n}p_i = 1$
	So when $p'_j = p_i = \frac{1}{n}$, we got maximun
	\begin{equation}
	\begin{aligned}
	\binom{n}{m}\prod_{j=1}^{m}{p'_j} &\leq \binom{n}{m}{(\frac{\sum_{j=0}^{m}{p'_j}}{m})^m}\\
		&\leq \binom{n}{m}{(\frac{\sum_{j=0}^{m}{\frac{1}{n}}}{m})^m}\\
		&= \binom{n}{m}(\frac{1}{n})^m\\
		&= \prod_{i=1}^{m-1}(1-\frac{i}{n})\\
		&\leq \prod_{i=1}^{m-1}e^{-\frac{i}{n}}\\
		&= e^{-\frac{m(m-1)}{2n}}
	<\frac{1}{2}
	\end{aligned}
	\end{equation}
	here we use the inequality $e^x \geq 1+x$.
	We got what we want
	$$m = \lceil \frac{1}{2}+\frac{1}{2}\sqrt{1+8n\ln{2}} \rceil$$
	So m above is the smallest m for which i can say that there would be at least one bin with at least 2 balls with probability $>\frac{1}{2}$

	\section{}
	Let $X_n$ denote the number of comparisons needs in sorting n elements. $Pr_n(X=k)$ denotes the probility for $X_n=k$.
	Use methods of Generating Functions, we suppose
	$$G_n(x)=\sum_{k}{Pr_n(X=k)x^k}$$
	As conditions, we have
	$$G_n(1)=\sum_{k}{Pr_n(X=k)}=1$$
	So
	$$E(X_n)=G'_n(1)$$
	If we choose one pivot from arrays, we divided the array into 2 sub-arrays. Thus
	$$G_n(x)=\frac{1}{n}x^{n-1}\sum_{j=1}^{n}G_{n-j}(x)G_{j-1}(x)$$
	Differentiating both sides of definition of $G_n(x)$ and differentiating again, we got
	\begin{equation}
	\begin{aligned}
		G_n'(x)&=\frac{1}{n}(n-1)x^{n-2}\sum_{j=1}^{n}G_{j-1}(x)G_{n-j}(x)\\
			&+\frac{1}{n}x^{n-1}\sum_{j=1}^{n}G'_{j-1}(x)G_{n-j}(x)\\
			&+\frac{1}{n}x^{n-1}\sum_{j=1}^{n}G_{j-1}(x)G'_{n-j}(x)\\
	\end{aligned}
	\end{equation}

	\begin{equation}
	\begin{aligned}
	G_n''(x)&=\frac{(n-1)(n-2)}{n}x^{n-3}\sum_{j=1}^{n}G_{j-1}G_{n-j}\\
			&+\frac{n-1}{n}x^{n-2}\sum_{j=1}^{n}G'_{j-1}(x)G_{n-j}(x)\\
			&+\frac{n-1}{n}x^{n-2}\sum_{j=1}^{n}G_{j-1}(x)G'_{n-j}(x)\\
			&+\frac{n-1}{n}x^{n-2}\sum_{j=1}^{n}G'_{j-1}(x)G_{n-j}(x)\\
			&+\frac{1}{n}x^{n-1}\sum_{j=1}^{n}G''_{j-1}(x)G_{n-j}(x)\\
			&+\frac{1}{n}x^{n-2}\sum_{j=1}^{n}G'_{j-1}(x)G'_{n-j}(x)\\
			&+\frac{n-1}{n}x^{n-2}\sum_{j=1}^{n}G_{j-1}(x)G'_{n-j}(x)\\
			&+\frac{1}{n}x^{n-2}\sum_{j=1}^{n}G'_{j-1}(x)G'_{n-j}(x)\\
			&+\frac{1}{n}x^{n-2}\sum_{j=1}^{n}G_{j-1}(x)G''_{n-j}(x)\\
	\end{aligned}
	\end{equation}
	We know that
	$$E(X_n)=G'_n(1)=M_n$$
	and
	$$G''_n(1)=(n-1)(n-2)+\frac{2}{n}(n-1)\sum_{j=1}^{n}M_{j-1}+\frac{2}{n}(n-1)\sum_{j-1}^{n}M_{n-j}
		+\frac{1}{n}\sum_{j=1}^{n}(G''_{j-1}(1)+G''_{n-j}(1))+\frac{2}{n}\sum_{j=1}^{n}M_{j-1}M_{n-j}$$
	Accoring to six preportities about Harmonic number,
	\begin{equation}
	\begin{aligned}
	\sum_{i=1}^n{H_i}&=(n+1)H_n-n\\
	\end{aligned}
	\end{equation}
	\begin{equation}
	\begin{aligned}
	\sum_{i=1}^{n}iH_{i-1}&=\frac{n(n+1)H_{n+1}}{2}-\frac{n(n+5)}{4}\\
	\end{aligned}
	\end{equation}
	\begin{equation}
	\begin{aligned}
	\sum_{i=1}^{n}i^2H_{i-1}&=\frac{6n(n+1)(2n+1)H_{n+1}-n(n+1)(4n+23)}{36}\\
	\end{aligned}
	\end{equation}
	\begin{equation}
	\begin{aligned}
	\sum_{i-1}^{n}i(n-i+1)H_{n-i}&=\frac{6nH_{n+1}(n^2+3n+2)-5n^3-27n^2-22n}{36}\\
	\end{aligned}
	\end{equation}	
	\begin{equation}
	\begin{aligned}
	\sum_{i=1}^{n}H_iH_{n+1-i}&=(n+2)(H^2_{n+1}-H_{n+1}^{(2)})-2(n+1)(H_{n+1}-1)\\
	\end{aligned}
	\end{equation}
	\begin{equation}
	\begin{aligned}
	H_{n+1}^{2}-H_{n+1}^{(2)}&=2\sum_{i=1}^{n}\frac{H_j}{j+1}
	\end{aligned}
	\end{equation}
	which, $H_n^{(2)}=\sum_{i=1}^{n}\frac{1}{i^2}$
	With equation(4)-(9)We got
	$$g''_n(1)=4(n+1)^2H_n^2-(16n+4)(n+1)H_n+\frac{2n(23n+17)}{2}-4(n+1)^2H_n^{(2)}$$
	Thus
	\begin{equation}
	\begin{aligned}
	Var(X_n)&=g''_n(1)+g'_n(1)-(g'_n(1))^2\\
		&=-4(n+1)^2H_n^{(2)}+7n^2-2(n+2)H_n+13n
	\end{aligned}
	\end{equation}
	As we know properity about $H_n^{(2)}$
	$$\lim_{n \to \infty}H_n^{(2)} = \frac{\pi^2}{6}$$
	So the variance of randomized quick sort is 
	$$Var(X_n)=O(n^2)$$

\end{document}














